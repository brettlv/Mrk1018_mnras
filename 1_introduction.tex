\section{Introduction}\label{sec:intro}
%Seyfert 1.9 (S1.9) is a Seyfert 1
%\textcolor{red}{}

\textcolor{red}{Type 1 and type 2 of Seyfert galaxies are classified based on the widths of optical spectral emission lines from the nuclei.} Type 1 Seyfert galaxies show both broad lines ($>$ 1000 $ \rm{km}\, \rm{s}^{-1}$) and narrow lines ($<$ 1000 $ \rm{km}\, \rm{s}^{-1}$), while type 2 Seyfert galaxies show only narrow lines. In the active galactic nucleus (AGN) unification model \citep[e.g.][]{1993ARA&A..31..473A}, A Seyfert 1  is face-on to the observer with broad lines visible to us, while Seyfert 2 is edge-on with broad line region blocked by the torus. The sub-classes (e.g. Seyfert 1.5, 1.8 and 1.9) are also introduced \citep[see ][]{1976MNRAS.176P..61O,1981ApJ...249..462O} based on the emission line width and relative strength of the broad-line to the narrow-line. For example, there are only broad H$\alpha$ lines in Seyfert 1.9 (S1.9), broad H$\alpha$ lines plus very weak broad H$\beta$ lines in Seyfert 1.8 (S1.8) and comparable H$\alpha$ and H$\beta$ lines in Seyfert 1.5 (S1.5). In recent years, several tens of changing-look AGNs (CL-AGNs hereafter) have been discovered within a timescale of decades or years \citep[e.g.][]{2014ApJ...796..134D,2014ApJ...788...48S,2015ApJ...800..144L,2016A&A...593L...8M,2016MNRAS.461.1927P,2019MNRAS.486..123R,2020ApJ...890L..29A,2020ApJ...901....1W,2020A&A...638A..91K}, or even months \citep[e.g.][]{2018arXiv181103694K,2019ApJ...883...94T}, which show disappearance/appearance of broad emission lines. It should be noted that the ``changing-look'' was originally used to describe the change of AGNs from Compton-thick to Compton-thin (or vice versa) based on the X-ray observations \citep[e.g.][]{2003MNRAS.342..422M}. 

The physical mechanisms of the changing-look phenomena are still under debate. On the one hand, the ``changing-look'' can be attributed to variable obscuration, such as obscuring material moving in or moving out from our line of sight \citep[e.g.][]{2013MNRAS.436.1615M,2014MNRAS.443.2862A,2015ApJ...815...55R,2018MNRAS.481.2470T}, where the intrinsic radiation spectrum is unchanged. On the other hand, the change may be triggered by the variation of intrinsic radiation \citep[e.g., the change of accretion disk,][]{1984MNRAS.211P..33P,2014MNRAS.438.3340E}. There are also many evidences for the variations in different wavebands in CL-AGNs, where the variations in radio \citep[e.g.][]{2016MNRAS.460..304K}, infrared \citep[e.g.][]{2017ApJ...846L...7S,2018ApJ...864...27S}, optical/ultraviolet \citep[e.g.][]{2019ApJ...885...44D} and X-ray \citep[e.g.][]{2016MNRAS.461.1927P,2019MNRAS.483L..88P,2020ApJ...898L...1R} bands are also reported when the AGNs change from type 1(2) to type 2(1). It is also found that the absorption is roughly unchanged in some CL-AGNs even the X-ray spectra varied a lot \citep[e.g.][]{2016A&A...593L...9H,2020ApJ...890L..29A,2020ApJ...901....1W}. Therefore, the intrinsic mechanism for the ``changing-look'' of AGNs is quite unclear. The high-cadence monitoring observations of the CL-AGNs can shed light on this issue. 

The Seyfert galaxy Mrk~1018 at $z=0.042$ has undergone a full cycle with twice type transitions during the past 40 years. It transited from S1.9 to S1 between 1979 and 1984 \citep{1986ApJ...311..135C} and returned to S1.9 after 30 years as a S1 \citep[see also][]{2016A&A...593L...8M,2016A&A...593L...9H,2017A&A...607L...9K}. The recent optical spectroscopic observations reveal that Mrk~1018 is a type 1 AGN in 2009 January, 2009 December and 2010 December and a type 1.9 AGN in 2015 January \citep{2016A&A...593L...8M,2018ApJ...861...51K}. The latest spectroscopic observation of Mrk~1018 in 2019 October shows a faint broad H$\beta$ line component in \citet{2020A&A...644L...5H}, which is similar to the type 1.9 spectrum as reported in \citet[][]{2016A&A...593L...8M}. Between 2010 and 2016, the optical and X-ray flux decrease by a factor of $\sim$ 17 and $\sim$ 8 \citep{2016A&A...593L...9H}, respectively, where the intrinsic absorption is negligible based on the X-ray observations. The spectroscopic observations during this period are absent, which may correspond to a changing-look phase.

The multi-wavelength and long-term variability of the CL-AGNs can improve us to understand the physical mechanisms for the CL behavior. In this work, we try to explore this issue based on performing an extensive data analysis for a CL-AGN of Mrk~1018 in radio band, optical/ultraviolet and X-ray bands. This paper is presented as as follows: In \autoref{sec:data}, we describe the observations and the data reduction in different wavebands. In \autoref{sec:result}, we present the multi-wavelength observational results. In \autoref{sec:discussion}, we discuss possible physics behind the observational results. Finally, we summarize our results in \autoref{sec:conclusion}. Throughout this work, we use a flat $\Lambda-$CDM cosmological model with $\Omega_{\rm{M}}$=0.27, $\Omega_\Lambda=$0.73 and a Hubble constant of 70 km s$^{-1}$ Mpc$^{-1}$. We adopt the luminosity distance $d_{\rm{L}}=176 $ Mpc and the black hole (BH) mass measurement $\log(M_{\rm{BH}}/M_{\odot})=7.84$ \citep{2017MNRAS.472.3492E,2018MNRAS.480.3898N} for Mrk~1018. 


%\textcolor{red}{Observations in different wavelengths have covered the timeline before and after the recent changing-look of Mrk~1018 for exploring the possible physical mechanism for the changing-look behavior.


%supports that the "changing-look" is driven by the activity of the central engine.
% 2016MNRAS.457..389M,2016ApJ...826..188R,2020MNRAS.491.4925G, quasar
% 2019ApJ...874....8M quasar candidate
 % 2018ApJ...864...27S, mid-infrared
 % 2019MNRAS.486..123R The reappearance of the broad emission lines
 % \citep[e.g.][]{}.





%\clearpage