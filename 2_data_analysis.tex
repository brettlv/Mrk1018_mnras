\section{Data reduction and analysis}\label{sec:data}
\subsection{X-ray data analysis}
We analyse the public archival data of \swift, \xmm, \chandra ~and \nustar during the period between 2005 and 2019. We use the cosmic abundances of \citet{2000ApJ...542..914W} and the photoelectric absorption cross sections from \citet{1995A&AS..109..125V}. All the X-ray spectra are fitted by an absorbed power-law model  \texttt{tbabs*zpowerlw} with the absorption by the Galactic hydrogen fixed at $N_{\rm{HI,Gal}}=2.43 \,\times \,10^{20}\, \rm{cm}^{-2}$ \citep[][]{2005A&A...440..775K} since no intrinsic absorption beyond Galactic is detected \citep[see][]{2016A&A...593L...9H,2017A&A...607L...9K}. The 2--10~keV flux are calculated by {\it cflux} component within {\tt XSPEC} (v12.10). The observation information and best-fitting parameters including photon index ($\Gamma$), unabsorbed flux in 2--10~keV ($F_{\rm{2-10~keV}}$) are listed in \autoref{tab:tablexray}. The long-term X-ray light curve in 2--10~keV band is shown in the top panel of \autoref{fig:multi-lc-secondaxis}.


\subsubsection{\xrt\,}
\label{data-xrt}
The X-ray telescope (XRT) on board \swift\, has the highest cadence monitoring observations of Mrk~1018 in the X-ray band, especially after 2015. We reprocess the archive data of \xrt\, observations performed in photon counting mode with {\scriptsize XRTPIPELINE}. The source region is a circle centered at the nucleus of Mrk 1018, the radius of which is determined by the count rate of each observation according to \citet{2009MNRAS.397.1177E}. We use {\scriptsize XSELECT} to extract the source and background spectra. The spectra are grouped by the a minimum of one count per bin. The XRT spectra of Mrk~1018 in the 0.5--10~keV range are fitted using the Bayesian X-ray Analysis (BXA)\footnote{\url{http://johannesbuchner.github.io/BXA/index.html}} method.  %'range' is right


\subsubsection{\chandra/ACIS-S}
We extract the ACIS-S spectra with CIAO (v4.12) and {\tt CALDB} (v4.9.1). For the observation in November of 2010 (ObsID 12868), which is affected by the pile-up effect, we adopt the fitting results from \citet{2016A&A...593L...9H} which excluded the bright pixels and corrected the photon loss for this observation. The other observations are extracted from a 3$\arcsec$ radius circle and the background spectra are extracted from an annulus with 5$\arcsec$ inner radius and 15$\arcsec$ outer radius \citep[see also][]{2017ApJ...840...11L}. Then the spectra are grouped by a minimum of 20 counts per bin and fitted in the 0.5--8~keV range. 



\subsubsection{$XMM-Newton$/EPIC-PN }
We analyse the archival data sets of Mrk~1018 derived by the EPIC-PN on board \xmm\,. The source is observed in 2005 (ObsID 201090201) and 2008 (ObsID 554920301). We reduce the PN data with {\scriptsize EPPROC} in \texttt{SAS-16.1.0}. The source and background regions are a 40$''$ and 60$''$ radius circle, respectively. Each spectrum is grouped by a minimum of 30 counts per bin and fitted in the 2--10~keV range. 


\subsubsection{$NuSTAR$}
We analyse the archival data sets of Mrk~1018 on board \nustar, which are reduced through the {\scriptsize NUPIPELINE} task of the {\scriptsize NUSTARDAS} package. The source region is a 50$''$ radius circle at the center of the source, and the background is extracted from the blank region. Each spectrum is grouped by a minimum of 30 counts per bin and fitted in the 3--79~keV range.

\subsection{\uvot\,data analysis}
\label{sec:uvot}
%f_uuu=(constants.c/(3465*units.AA)).to(u.Hz).value
%f_uw1=(constants.c/(2600*units.AA)).to(u.Hz).value
%f_um2=(constants.c/(2246*units.AA)).to(u.Hz).value
%f_uw2=(constants.c/(1928*units.AA)).to(u.Hz).value
There are six filters in the optical/UV band of \uvot, which are V, B, U, UVW1, UVM2, and UVW2 bands. We use the tool \textit{uvotsource} to do the aperture photometry for each filter of all the observations. The source aperture radius is 5$\arcsec$ and the background is chosen in a blank region with a much larger radius. According to the results in \citet{2018MNRAS.480.3898N}, the emission of the host galaxy is dominated in the V and the B band, we then discard all the results of the V and the B band. 

In order to correct the Galactic extinction, we adopt $E(B-V) = 0.036$ \citep[see][]{2018MNRAS.480.3898N} and $R_{V}=3.1$ for the Galactic extinction and calculate the values of $A_{\lambda}$ for U, UVW1, UVM2, and UVW2 band are  0.18, 0.25, 0.35 and 0.31 assuming the extinction model of \citet{2007ApJ...663..320F}. In order to estimate the intrinsic optical/UV flux from the nucleus, we subtract the contribution from the host galaxy, which is estimated from the broadband spectral modeling result in \citet{2018MNRAS.480.3898N}. The results of the long-term optical and UV light curves from \uvot\, are shown in the middle panel of \autoref{fig:multi-lc-secondaxis} and listed in \autoref{tab:tableuvot}.




\subsection{Radio data analysis}
\label{subsec:vla}
%\subsubsection{VLA}
We analyse the archival data of Very Large Array (VLA) observations for Mrk~1018 with \textsc{casa} version 5.3.0 \citep{2007ASPC..376..127M}. For the reduction of the pre-EVLA upgrade VLA data (VLA, project ID: AU0020, AB0476, AB0540, and AB0878), we manually flag and calibrate the data, then clean the image following the instruction\footnote{\url{https://casaguides.nrao.edu/index.php/VLA_5_GHz_continuum_survey_of_Seyfert_galaxies}}. For the new Karl G. Jansky Very Large Array data (JVLA, project ID: 16A-444, 16B-084, and 18B-245), calibrations are performed using script {EVLA\_pipeline1.4.2}\footnote{\url{https://science.nrao.edu/facilities/vla/data-processing/pipeline/scripted-pipeline}}. Different bands are split into different MS files after checking the radio frequency interference and calibration. Then the source is imaged using {\scriptsize TCLEAN} method and integrated flux is estimated via \textsc{imfit} task. The uncertainty of flux density is calculated from $\sigma_\mathrm{S}=\sqrt{(rms)^2+(0.05\times S)^2}$, where $5 \%$ absolute flux error is taken into account, except for the quick look image result in epoch 1 of {\em VLA Sky Survey (VLASS1.1)}, where $15 \%$ system error is considered according to the {\em VLASS Epoch 1 Quick Look Users Guide} \footnote{\url{https://science.nrao.edu/science/surveys/vlass/vlass-epoch-1-quick-look-users-guide}}. The imaging results are listed in Table.~\ref{tab:tableradio}. 


To compare between different periods and keep consistent with the radio and X-ray correlation in literature, we convert the radio flux to 5 GHz if it was observed at other wavebands using $S_v \propto v^{-\alpha_\mathrm{R}}$. The radio flux density of Mrk~1018 in the same band varied little before 2015, so we assume the radio spectral index also remains constant during this period. We calculate the $\alpha_\mathrm{L-X} =0.3 \pm 0.08$ using the observations on 50970 MJD (X band) and 52490 MJD (L band), then convert the flux densities at other bands to 5 GHz during this period (see \autoref{tab:tableradio}). There is one observation with two bands (C and X) available on MJD 57481, the $\alpha_\mathrm{C-X} =0.25\pm0.1$ is consistent within uncertainties with previous measured $\alpha_\mathrm{L-X}$. We then use this value to convert the flux densities of other bands to 5 GHz after MJD 57481. The estimated radio light curve at 5 GHz after 2005 is present in the bottom panel of \autoref{fig:multi-lc-secondaxis}.

