\section{Discussion}\label{sec:discussion}
\subsection{The spectral evolution and possible accretion mode transition }
\label{sec:spectral evolution}
The X-ray photon index can shed light on the physical properties of the hot plasma in the advection-dominated
accretion flow (ADAF) or corona above and below the cold disk, which is controlled mainly by the electron temperature and optical depth. The negative/positive correlation of $\Gamma$-$\log{L_\mathrm{X}/L_\mathrm{Edd}}$ below/above a critical value ($L_\mathrm{X}/L_\mathrm{Edd} \sim 10^{-3}$) has been reported in both AGNs and X-ray binaries (XRBs) \citep[e.g.][]{2008ApJ...682..212W,2009MNRAS.399..349G,2011A&A...530A.149Y,2015MNRAS.447.1692Y,2020ApJ...889L..18Y}. The reason for the opposite X-ray spectral behavior is thought to be the differences of the seed photons for the Compton scattering, i.e. the seed photons are from the synchrotron emission of the hot ADAF at the lower luminosity branch, while from the thermal emission from Shakura--Sunyaev disk \citep[SSD; e.g. ][]{2013ApJ...764....2Q} at the higher luminosity branch. There is an evident negative correlation of $\Gamma$-$\log{L_\mathrm{X}/L_\mathrm{Edd}}$ in the type 1.9 phase of Mrk~1018, which is similar to the low-luminosity AGNs and the low/hard state XRBs. The data in the type 1 phase deviate from the negative correlation evidently, where the physical origin for optical and X-ray emission should be changed. 

The optical/UV-to-X-ray spectral index \alphaox\, is a good indicator of the broadband spectral energy distribution (SED), where the optical/UV and X-ray emission come from the cold SSD (or outer truncated SSD) and hot corona/ADAF respectively in AGNs. The negative and positive correlations of \alphaox-$\log{L_\mathrm{X}/L_\mathrm{Edd}}$ have been found in low-luminosity AGNs \citep[e.g.][]{2011ApJ...739...64X,2017MNRAS.471.2848L} and luminous AGNs \citep[e.g.][]{2010A&A...512A..34L, 2013A&A...550A..71V,2016ApJ...819..154L}. The ADAF model can roughly explain the negative \alphaox\,--$L_\mathrm{bol}$ correlation at the lower luminosity branch \citep{2011ApJ...739...64X,2017MNRAS.471.2848L}, and the disk-corona model can explain the positive \alphaox\,--$L_\mathrm{bol}$ correlation at the higher luminosity branch \citep{2017A&A...602A..79L, 2018MNRAS.480.1247K,2019A&A...628A.135A}. \citet{2011MNRAS.413.2259S} simulated the spectral states of AGNs by analogy with BHXRBs and found that the simulated AGNs at different spectral states and luminosity roughly follow a ``V''-shape \alphaox--$L_\mathrm{bol}$ correlation \citep[negative/positive correlation below/above a critical value, see also in ][]{2019ApJ...883...76R}. In other words, the opposite correlations between \alphaox\, and $L$ also support the idea that the accretion mode changes in the high and low luminosity AGNs \citep[see][]{2011MNRAS.413.2259S,2019ApJ...883...76R}.
There is an evident negative correlation of \alphaox-$\log{L_\mathrm{X}/L_\mathrm{Edd}}$ in the type 1.9 AGN phase of Mrk~1018, which is similar to the low-luminosity AGNs. The data in type 1 phase apparently deviate from the negative correlation, which indicates that the accretion mode is different from type 1.9 phase. A much stronger SSD component in type 1 phase will lead to a higher \alphaox. 

The evolutions of both $\Gamma$-$\log{L_\mathrm{X}/L_\mathrm{Edd}}$ and \alphaox-$\log{L_\mathrm{X}/L_\mathrm{Edd}}$ suggest that the strong evolution of underlying accretion disk in Mrk 1018 during the type transition. The change of the broad emission lines is most possibly regulated by the appearance or disappearance of cold disk near the black hole horizon, which regulates the radiative efficiency of accretion flow and/or ionization luminosity for the clouds in broad line region. 



\subsection{The variability timescale}
\label{sec:timescale}
In Mrk 1018, the broadband spectra are dramatically changed with the rapid decay of luminosity \citep[see also ][]{2016A&A...593L...9H,2018MNRAS.480.3898N} between 2010 and 2015.
More and more studies show that the spectral evolution of CL-AGNs somehow are quite similar to the spectral state transitions in Galactic BH XRBs \citep{2018MNRAS.480.3898N,2019arXiv190904676R,2019ApJ...883...76R,2020MNRAS.492.2335L}. The state transitions in BHXRBs normally occur on timescale of days to tens of days \citep{2009ApJ...701.1940Y,2010MNRAS.403...61D}, which is usually
attributed to the viscous timescale of the inner disk in the truncated accretion disk model \citep[see reviews in ][]{2007A&ARv..15....1D}. The viscous timescale $\tau_\mathrm{vis}$ for a $10^{8} M_{\odot}$ AGN is given by,
\begin{equation}
\tau_\mathrm{vis} \sim 5.7\times 10^{-3} \alpha^{-1}(\frac{M_\mathrm{BH}}{10^8M_{\odot}})(\frac{R_\mathrm{tr}}{R_g})^{3/2} (\frac{H}{R})^{-2} \, \mathrm{days} 
\end{equation} 
 The $H/R$ is the ratio between disk height and disk radius, the $R_\mathrm{tr}$ is the inner truncated radius. So if the truncated radius of the accretion disk is smaller than  $\sim$20$R_{g}$, the viscous timescale will roughly agree with the spectral evolution timescale ($\lesssim$5 years) of Mrk 1018 for $\alpha=0.1$ and $H/R=0.05$ \citep[e.g.][]{2020MNRAS.492.2335L}. Such a viscous timescale is too short for a standard disk, since the required $H/R$ is much larger than that expected in standard disk. So there are also some different scenarios proposed to account for such a short timescale, such as the thermal and heating/cooling front timescales in the innermost regions of the accretion disk \citep{2018ApJ...864...27S}, a radiation-pressure dominated accretion disk with faster sound speed and hence the viscous speed \citep{2018MNRAS.480.3898N}, a thick disk supported by the magnetic pressure \citep{2019MNRAS.483L..17D}, and the radiation pressure instability and transition between standard disk and ADAF \citep{2020A&A...641A.167S}. 
 
A re-flare with a timescale on a magnitude of hundreds of days during the decay phase was found, where the physical reason for the variability is unclear. The rise timescale ($\leq$ 100 days) of the re-flare is consistent with the thermal timescale.  A short ``turn-on'' (appearance of broad emission lines and increasing luminosity by a factor of 8) timescale $\sim$ 70 days of PS1-13cbe is also suggested to be compatible with the thermal timescale $\sim$ 99 days under the scenario of UV/X-ray reprocessing \citep[][]{2019MNRAS.487.4057K}. However, there is no information about the emission line of Mrk 1018 during the re-flare. 

The time lags on a magnitude of several days between X-ray and optical/UV flux variations have been studied in several AGNs, such as NGC 4151 ($\sim$ 3--4 days), NGC 5548 ($\sim$ 2--5 days), NGC 3783 ($\sim$ 3--9 days), Ark 120 ($\sim$ 10--12 days), \citep[e.g. ][]{2009MNRAS.397.2004A,2017ApJ...840...41E,2019ApJ...870..123E,2020MNRAS.494.1165L} where X-ray leads the optical/UV variations and Mrk 509 ($\sim$ 2--5 days) \citep[e.g. ][]{2019ApJ...870..123E} where X-ray lags the optical/UV. The behavior challenges the standard reprocessing picture. The estimated time lag $\tau \sim$ 20 days of optical/UV behind X-ray variations of Mrk 1018 during the type 1.9 phase in 2018 is longer than that expected from standard accretion disk \citep[see also][]{2018ApJ...854..107F,2020MNRAS.494.1165L}. The physical mechanism for this time lag is unclear.  The time lag $\tau \sim 20$ days corresponds to a light travel distance of $\sim$ 5000 $R_g$. The BLR scale ($R_{\rm{BLR}}$) of Mrk~1018 is inferred as $\sim$ 5000 $R_g$ (H$\alpha$ line width 4000 $\rm{km}\, \rm{s}^{-1}$) and $\sim$ 7400 $R_g$ (H$\alpha$ line width 3300 $\rm{km}\, \rm{s}^{-1}$) \citep[see ][]{2016A&A...593L...8M} following the $M-\sigma$ relation in \citet[][]{2015ApJ...801...38W}. The time lag of optical/UV behind X-ray might be produced through the X-ray reprocessing at a larger distance in an optically thick disk scenario or corresponding to the BLR \citep[e.g.][]{2009MNRAS.397.2004A}.

 
\subsection{Radio--X-ray correlation}
The simultaneous X-ray and radio emission track the connection between accretion and ejection activities of CL-AGNs \citep[e.g.][]{2016MNRAS.460..304K,2021MNRAS.tmp..700Y}.
It is widely accepted that there is a non-linear correlation between the 5 GHz radio luminosity and the 2--10 keV X-ray luminosity spanning from the super-massive black holes to stellar mass black holes: $L_\mathrm{R}\propto L_\mathrm{X}^{0.6}$ \citep[e.g.][]{2003MNRAS.345.1057M,2004A&A...414..895F}. There are also some sources following a steeper radio--X-ray correlation with a power-law index of $\sim$1.4 at higher luminosity, a flat radio--X-ray correlation with a power-law index of $\sim$0 at transition stage \citep[e.g.][]{2011MNRAS.414..677C,2014ApJ...788...52C,2016MNRAS.463.2287X}. The ``hybrid'' radio--X-ray correlation is possibly regulated by the underlying accretion processes \citep[e.g.][]{2016MNRAS.456.4377X} or due to different jet properties \citep[e.g.][]{2018MNRAS.481.4513I,2018MNRAS.473.4122E}. It should be noted that Mrk~1018 roughly follows the radio--X-ray correlation as defined by other BH sources \citep{2012MNRAS.419..267P} even though its own correlation is quite flat. The X-ray luminosity's decline by a factor $\sim $7.5 requires a variability by a factor of $\sim$ 3.3 in radio luminosity to follow the radio--X-ray correlation with a slope of $0.6$, which is far beyond the radio variability of Mrk~1018. The possible physical reasons include: 1) the radio variability timescale is much longer than X-ray since that the radio emission comes from a larger scale of jet; 2) the flat radio--X-ray correlation is mainly regulated by the X-ray emission, where the accretion rate does not vary much but the radiative efficiency change a lot when the accretion rate is close to a critical value. The Eddington ratio of bolometric luminosity is around 1\% \citep[][]{2018MNRAS.480.3898N}, which suggests that the radiative efficiency in Mrk~1018 may indeed easily suffer strong variations due to the transition of accretion modes. 
  




