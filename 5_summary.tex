\section{Summary}
\label{sec:conclusion}
The main results are summarized as follows,
\begin{enumerate}

\item We present the long-term and multi-wavelength variability from radio to X-ray band for a CL-AGN of Mrk 1018. We find a re-flare in both optical/UV and X-ray bands during the decay phase and a time lag $\sim$ 20 days of optical/UV behind X-ray emission during the type 1.9 phase.    

\item We find negative correlations of $\Gamma$-$L_\mathrm{X}/L_\mathrm{Edd}$ and \alphaox-$L_\mathrm{X}/L_\mathrm{Edd}$ in the type 1.9 phase, which are consistent with the prediction of the radiatively inefficient accretion (e.g., ADAF). The data in the type 1 phase deviate from the negative correlations, where the accretion mode may change into a radiatively efficient disk-corona system. Therefore, the change of broad emission lines might be regulated by the underlying accretion process. 

\item The radio emission in CL-AGN of Mrk~1018 was roughly unchanged before the type 1.9 phase (2005--2015), then it slightly declined about $\sim$ 20\% during 2016--2017. Therefore, radio--X-ray correlation is quite flat in Mrk 1018, which is much shallower than that found in low-luminosity AGNs and low-hard state XRBs with a slope of $\sim 0.6$. 

\end{enumerate}

 