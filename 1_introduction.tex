\section{Introduction}\label{sec:intro}

Type 1 and type 2 active galactic nuclei (AGNs) are classified based on the widths of optical spectral emission lines. Type 1 AGNs show both broad lines ($>$ 1000 $ \rm{km}\, \rm{s}^{-1}$) and narrow lines ($<$ 1000 $ \rm{km}\, \rm{s}^{-1}$), while type 2 AGNs show only narrow lines. In the AGN unification model \citep[e.g.][]{1993ARA&A..31..473A}, type 1 AGNs are viewed face-on with the broad-line region (BLR) visible to the observer, while type 2 AGNs are viewed edge-on with the broad-line region blocked by a putative dusty torus. The sub-classes (e.g., type 1.5, 1.8, and 1.9) are also introduced \citep[see ][]{1976MNRAS.176P..61O,1981ApJ...249..462O} based on the emission-line width and relative strength of the broad-line to the narrow-line. There are broad H$\alpha$ line and very weak or undetectable broad H$\beta$ line in type 1.9, broad H$\alpha$ line and weak broad H$\beta$ line in type 1.8 \citep[see][]{1986ApJ...311..135C}, and comparable H$\alpha$ and H$\beta$ lines in type 1.5.  

In recent years, several tens of so-called changing-look AGNs (CL-AGNs hereafter) have been discovered,  which show disappearance/appearance of broad emission lines within a timescale of decades or years \citep[e.g.][]{2014ApJ...796..134D,2014ApJ...788...48S,2015ApJ...800..144L,2016A&A...593L...8M,2016MNRAS.461.1927P,2016ApJ...826..188R,2018ApJ...862..109Y,2019MNRAS.486..123R,2020MNRAS.491.4925G,2020ApJ...890L..29A,2020ApJ...901....1W,2020A&A...638A..91K}, or even months \citep[e.g.][]{2019MNRAS.487.4057K,2019ApJ...883...94T}. The term ``changing-look'' was originally used to describe the changes of AGNs from Compton-thick to Compton-thin (or vice versa) based on the X-ray observations \citep[e.g.][]{2003MNRAS.342..422M}. In this paper, the term ``changing-look" refers to the change in optical emission lines. The physical mechanism of the changing-look phenomena is still under debate. On the one hand, the ``changing-look'' can be attributed to variable obscuration, such as obscuring material moving in or out from our line of sight \citep[e.g.][]{2013MNRAS.436.1615M,2014MNRAS.443.2862A,2015ApJ...815...55R,2018MNRAS.481.2470T}, where the intrinsic emission is roughly unchanged. On the other hand, the changing-look may associate with the variation of intrinsic radiation \citep[e.g., the change of accretion disk,][]{1984MNRAS.211P..33P,2014MNRAS.438.3340E}. Correlated variability in the X-ray \citep[e.g.][]{2016MNRAS.461.1927P,2019MNRAS.483L..88P,2020ApJ...898L...1R}, the optical/UV \citep[e.g.][]{2019ApJ...885...44D} and the infrared band \citep[e.g.][]{2017ApJ...846L...7S,2018ApJ...864...27S} with the AGN type change supports variable accretion as the physical origin in some CL-AGNs, which is also supported by that the absorption is roughly unchanged in some CL-AGNs \citep[e.g.][]{2016A&A...593L...9H,2020ApJ...890L..29A,2020ApJ...901....1W}. 

 
The Seyfert galaxy Mrk~1018 at $z=0.042$ has undergone a full cycle with twice types transitions during the past 40 years. It transited from type 1.9 to type 1 between 1979 and 1984 \citep{1986ApJ...311..135C} and returned to type 1.9 after 30 years \citep[see][]{2016A&A...593L...8M,2016A&A...593L...9H,2017A&A...607L...9K}. The optical spectroscopic observations reveal that Mrk~1018 is a type 1 AGN in 2010 December and a type 1.9 AGN in 2015 January \citep{2016A&A...593L...8M,2018ApJ...861...51K}. The optical spectroscopic observation of Mrk~1018 in 2019 October shows a faint broad H$\beta$ line component in \citet{2020A&A...644L...5H}, which is similar to the type 1.9 spectrum as reported in \citet[][]{2016A&A...593L...8M}. 

In this work, we explore the possible physical mechanism for the changing look by performing an extensive data analysis for a CL-AGN of Mrk~1018 in the radio, the optical/UV, and the X-ray bands. We mainly focus on the period of 2005-2019, which is covered by multi-wavelength observations. According to the optical spectroscopic results, we identify that the period of 2005--2010 is the type 1 AGN phase, and 2015--2019 is the type 1.9 AGN phase. The paper is organized as follows. In \autoref{sec:data}, we describe the observations and the data reduction in different wavebands. In \autoref{sec:result}, we present the multi-wavelength observational results. In \autoref{sec:discussion}, we discuss possible physics behind the observational results. Finally, we summarize our results in \autoref{sec:conclusion}. Throughout this work, we use a flat $\Lambda-$CDM cosmological model with $\Omega_{\rm{M}}$=0.27, $\Omega_\Lambda=$0.73 and a Hubble constant of 70 km s$^{-1}$ Mpc$^{-1}$. We adopt the luminosity distance $d_{\rm{L}}=176 $ Mpc and the black hole (BH) mass measurement $\log(M_{\rm{BH}}/M_{\odot})=7.84$ \citep{2017MNRAS.472.3492E,2018MNRAS.480.3898N} for Mrk~1018. 


