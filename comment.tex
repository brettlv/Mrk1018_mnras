\begin{comment}
picurl=https://github.com/brettlv/brettlv.github.io/pythoncode/Mrk1018/pic
This is a simple template for authors to write new MNRAS papers.
See \texttt{mnras\_sample.tex} for a more complex example, and \texttt{mnras\_guide.tex}
for a full user guide.
All papers should start with an Introduction section, which sets the work
in context, cites relevant earlier studies in the field by \citet{Others2013},
and describes the problem the authors aim to solve \citep[e.g.][]{Author2012}.
\end{comment}
\begin{comment}
Normally the next section describes the techniques the authors used.
It is frequently split into subsections, such as Section~\ref{sec:maths} below.
\end{comment}


\begin{figure*}
\centering
	% To include a figure from a file named example.*
	% Allowable file formats are eps or ps if compiling using latex
	% or pdf, png, jpg if compiling using pdflatex
	\includegraphics[width=0.9\textwidth]{./pic/subplots-xrt_uvot-radio-second-right-part.png}
    \caption{Light curves of Mrk~1018 in X-ray and UV band with high-cadence monitoring observations between 58350 and 58450 MJD.}
    \label{fig:x-ray-uv-lc-rp-secondaxis}
\end{figure*}

\begin{figure*}
\centering
	% To include a figure from a file named example.*
	% Allowable file formats are eps or ps if compiling using latex
	% or pdf, png, jpg if compiling using pdflatex
	\includegraphics[width=0.9\textwidth]{./pic/subplots-radio-second_freq.png}
    \caption{The radio light curve of Mrk~1018. Filled markers represent the observation flux at corresponding frequency and blank circles represent the radio flux scaled to 5 GHz. Red and blue vertical dashed lines represent the time of optical spectroscopic confirmation at type 1 and type 1.9, respectively.}
    \label{fig:radio-lc}
\end{figure*}



Region between them is marked as changing-look phase. Grey shallow region shows that the type transition occurred between 2013 and 2015 inferred by \citet{2017A&A...607L...9K} due to the rapid dimming of X-ray flux. Horizontal dotted line in the top panel represents that X-ray flux keeps almost constant until around 2009. The inclined dotted lines in the top and middle panel show the fitting of light curve in X-ray and UV band between 2008 and 2015 with characteristic \textit{e}-folding decay timescale $\tau$. In faint state after 2016, there is rapid variability in X-ray and UV band (see also \autoref{fig:x-ray-uv-lc-rp-secondaxis} for details). At the bottom panel, the label ``$F_\mathrm{5\,GHz}$'' marked with blank circles represent the radio flux scaled to 5 GHz. The whole radio light curve including the period before 2005 is shown in \autoref{fig:radio-lc}.




\begin{figure}
\centering
	% To include a figure from a file named example.*
	% Allowable file formats are eps or ps if compiling using latex
	% or pdf, png, jpg if compiling using pdflatex
	\includegraphics[width=0.45\textwidth]{./pic/Mrk1018_Luvot_L2_10_correlation-fig-without-outlier_clip.png}
    \caption{The $L_{\mathrm{UV}}$-$L_{\mathrm{2-10\,keV}}$ correlation before MJD 57033.}   
    \label{fig:luv_lx}
\end{figure}



\subsection{What causes the changing-look?}
\label{sec:Changing-look}


\begin{figure}
\centering
	% To include a figure from a file named example.*
	% Allowable file formats are eps or ps if compiling using latex
	% or pdf, png, jpg if compiling using pdflatex
	\includegraphics[width=0.45\textwidth]{./pic/Mrk1018_2individuals_LX_LUV.png}
    \caption{The $L_{\mathrm{UVW1}}$-$L_{\mathrm{X}}$ correlation.}   
    \label{fig:luv_lx}
\end{figure}


Mrk~1018's negative \alphaox-$\log{L_\mathrm{X}/L_\mathrm{Edd}}$ correlation in type 1.9 is apparently different from that in type 1. Applying the simplified expression $\log{\frac{2-\Gamma}{5^{2-\Gamma}-1}}\approx 0.35\times \Gamma -0.91$, we can rewrite \autoref{definition_alpha_ox} as
\begin{equation}
\frac{\alpha_\mathrm{OX}}{0.384} = {\log L_\mathrm{UVW1} -\log L_\mathrm{2-10~ keV}-0.35\times\Gamma+const}
\label{alpha_ox_re}
\end{equation} when $\Gamma \neq 2$.
Combining the slope $k=-0.87$ of the $\Gamma$-$\log{L_\mathrm{X}}$ correlation and the slope $m=-0.24$ of the \alphaox-$\log{L_\mathrm{X}}$ correlation, we can derive that $\log L_\mathrm{UVW1} = 0.07\times \log L_\mathrm{X} + const$ from \autoref{alpha_ox_re} in the type 1.9 phase. This result is consistent with the non-correlation between $\log{L_\mathrm{X}}$ and $\log{L_\mathrm{UV}}$ in type 1.9 but differs from the changing-look phase and the AGN samples. In both optically and X-ray selected AGN samples, the relationship between $\log{L_\mathrm{UV}}$ and $\log{L_\mathrm{X}}$ (parameterized as $\log L_\mathrm{X}=  \gamma_1 \log L_\mathrm{UV} +\beta_1$) exhibits a slope $\gamma_1$ around $0.5-0.7$ \citep[e.g.][and references therein]{2017A&A...602A..79L}, which derives the slope $\gamma$ around $1.4-2$ of $\log L_\mathrm{UV}=\gamma \log L_\mathrm{X}+\beta$. The slope $\gamma=1.7$ of $\log L_\mathrm{UVW1}=\gamma \log L_\mathrm{X}+\beta$ during the changing-look phase is consistent with the samples. Mrk~1018's $\log L_\mathrm{UV}$ follows different relationships with $\log L_\mathrm{X}$ in different types.






















\section*{Data Availability}
The inclusion of a Data Availability Statement is a requirement for articles published in MNRAS. Data Availability Statements provide a standardised format for readers to understand the availability of data underlying the research results described in the article. The statement may refer to original data generated in the course of the study or to third-party data analysed in the article. The statement should describe and provide means of access, where possible, by linking to the data or providing the required accession numbers for the relevant databases or DOIs.