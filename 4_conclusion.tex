\section{Discussion}\label{sec:discussion}
\subsection{The spectral evolution and possible accretion mode transition }
\label{sec:spectral evolution}
\textcolor{red}{The hot accretion flow such as the Accretion Dominated Accretion Flow (ADAF) is thought to dominate the broad band emission from optical to X-ray \citep[see reviews in ][]{2014ARA&A..52..529Y} in low-luminosity AGNs (LLAGNs), while the optical/UV emission originates from the Shakura–Sunyaev disk \citep[SSD; e.g. ][]{2013ApJ...764....2Q} in luminous AGNs. The X-ray emission of AGNs is mainly comes from the Compton scattering in the hot ADAF or corona that stay below/above the cold SSD \citep[e.g.][]{1991ApJ...380L..51H}. The X-ray photon index reflect the properties of the hot corona, which is controlled by the electron temperature and optical depth of hot plasma(ADAF or corona).} The negative/positive correlation of $\Gamma$-$\log{L_\mathrm{X}/L_\mathrm{Edd}}$ below/above a critical value ($L_\mathrm{X}/L_\mathrm{Edd} \sim 10^{-3}$) has been found in both AGNs and BH XRBs samples \citep[e.g.][]{2008ApJ...682..212W,2009MNRAS.399..349G,2011A&A...530A.149Y,2015MNRAS.447.1692Y}. There is an evident anti-correlation of $\Gamma$-$\log{L_\mathrm{X}/L_\mathrm{Edd}}$ in the type 1.9 phase of Mrk~1018, which is similar to the low-luminosity AGNs and the low/hard state XRBs. The data in type 1 phase deviate the anti-correlation evidently. They may belong to the positive branch where the observational data are still limited. Mrk~1018 shares almost the same value of $L_\mathrm{X}/L_\mathrm{Edd}\sim 10^{-3}$ as the pivoting line of the negative/positive correlations of $\Gamma$-$\log{L_\mathrm{X}/L_\mathrm{Edd}}$ with the AGNs and the XRBs samples. The negative and positive correlations were attributed to ADAF and SSD-corona models, respectively. 






A flare was found during the decay phase, where the physical reason is still unclear. 


\textcolor{red}{During the re-flare, the spectrum becomes very hard ($\Gamma \sim 1.4$) as the X-ray luminosity increases, which could be explained by the increase of seed photons for comptonization causes the increase of optical depth. Our results suggests the type 1.9 is associated with the ADAF and the type 1 is associated with the SSD. The disappearance of the inner disk might be the explanation for the changing look.} The change of accretion mode is also suggested in other CL-AGNs \citep{2020ApJ...890L..29A} and highly variable AGN \citep{2020MNRAS.492.2335L}.  

The negative and positive correlations of \alphaox-$\log{L_\mathrm{X}/L_\mathrm{Edd}}$ have also been found in the low-luminosity AGNs\citep[e.g.][]{2011ApJ...739...64X,2017MNRAS.471.2848L} and luminous AGNs samples \citep[e.g.][]{2010A&A...512A..34L, 2013A&A...550A..71V,2016ApJ...819..154L}. \citet{2011MNRAS.413.2259S} simulated the spectral states of AGNs by analogy with BHXRBs, and found that the simulated AGNs at different spectral states and luminosity roughly follow a V-shape \alphaox--$L_\mathrm{bol}$ correlation (negetive/positive correlation below/above an critical value) \citep[see also in ][]{2019ApJ...883...76R}. \citet{2019arXiv190904676R} finds two CL-AGNs roughly follow a V-shape \alphaox--$L_\mathrm{UV}$ correlation when the two CL-AGNs are type 2 and type 1, respectively. \textcolor{red}{The ADAF model} can roughly explain the negative \alphaox\,--$L_\mathrm{bol}$ correlation at the lower luminosity branch \citep{2011ApJ...739...64X,2017MNRAS.471.2848L}, and the disk-corona model can explain the positive \alphaox\,--$L_\mathrm{bol}$ correlation at the higher luminosity branch \citep{2017A&A...602A..79L, 2018MNRAS.480.1247K,2019A&A...628A.135A}. There is an evident anti-correlation of \alphaox-$\log{L_\mathrm{X}/L_\mathrm{Edd}}$ in the type 1.9 AGN phase of Mrk~1018, which is similar to the low-luminosity AGNs. The pivoting line of the negative/positive correlations of \alphaox-$\log{L_\mathrm{bol}/L_\mathrm{Edd}}$ is roughly at $L_\mathrm{bol}/L_\mathrm{Edd} \sim$ $10^{-3}$, assuming that $L_\mathrm{bol}/L_\mathrm{X} \sim$ $30$ \citep[see][]{2011ApJ...739...64X,2017MNRAS.471.2848L}.  The anti-correlation of \alphaox-$\log{L_\mathrm{X}/L_\mathrm{Edd}}$ in the type 1.9 and the evident deviation in the type 1 also supports the accretion mode is different in the type 1 and type 1.9 AGN phase for Mrk~1018.     

The physical reason for triggering the changing-look is still unclear. Some models of the broad line region (BLR) \citep[see a recent review in ][and references therein]{2019OAst...28..200C} related with the SSD can be account for the type change in the CL-AGNs. For example, in the disk-wind model for the BLR origin, the AGNs evolve from type 1 to type 1.9 as the luminosity decreases \citep[see][]{2014MNRAS.438.3340E}. \citet{2018MNRAS.480.3898N} suggests that the soft X-ray excess contributes most ionizing photons, and the drop of soft X-ray excess causes the disappearance of broad emission line \citep[see also in ][]{2020MNRAS.492.2335L}. \textcolor{red}{ It has been observed that the broad line luminosity is positively correlated with the UV luminosity, and the broad line width is negatively correlated with the UV luminosity \citep[e.g.][]{2019ApJ...885...44D}. When the UV luminosity decreases, the broad line emission would decrease and the line width would broaden to be difficult to detect. We did detect the decay in both the X-ray and UV bands during the changing-look.} If there is a critical luminosity and it is below the peak of the re-flare, then Mrk~1018 would change its type back and forth between type 1 and type 1.9 during the re-flare. The UV luminosity only varied by a factor of $\sim$1.4 and the X-ray luminosity varied by a factor of $\sim$3.2 at the peak of the re-flare. If the critical luminosity is above the peak of re-flare, then the changing-look would happen in earlier timeline before 2013 and the changing-look would be more likely associated with the decline of UV luminosity since the peak X-ray luminosity of the re-flare is close to the X-ray luminosity in type 1. The optical confirmation would help to find out which one directly influence the type change.



\subsection{The variability timescale}
\label{sec:timescale}
The flux variations of AGNs on different timescales and different wavelengths have been studied for a long time \citep[see reviews in ][]{1997ARA&A..35..445U}. We calculate the \textit{e}-folding decay timescale is $\sim$ 1000 days  during the rapid decay in both the X-ray and UV bands, which is consistent with the changing-look timescale of less than 5 years (between 2010 Nov. and 2015 Jan.). The viscous timescale at typical outer disk radius of $R_\mathrm{disk}$ around hundreds of $R_g$ is much longer than the changing-look timescale \citep[see also][]{2012MmSAI..83..469L,2018MNRAS.475.1190Y}. \textcolor{red}{The changing-look timescale might correspond the timescale of the evaporation of the inner disk \citep{}.} The thermal timescale is considered as the origin of the short-term (tens of days) variability in CL-AGNs. The rise timescale $\sim$ 98 days of the re-flare is consistent with the thermal timescale assuming R $\sim$ 200 $R_g$. Turn-on for CL-AGN could happen within timescale as short as $\sim$70 days\citep[e.g.][]{2019MNRAS.487.4057K}, which is suggested to be mostly compatible with the thermal timescale of $\sim$ 99 days assuming an optical emission distance of R $\sim$ 400 $R_g$. But we cannot draw a conclusion that the type change happened during the re-flare within such short timescale for Mrk~1018.


\subsection{R-X correlation}
It is widely accepted that there is a non-linear correlation between the 5 GHz radio luminosity and the 2--10 keV X-ray luminosity spanning over different mass black holes from the super-massive black holes to stellar mass black holes. The $L_\mathrm{R}$ is proportional to $L_\mathrm{X}^{0.6}$ \citep{2003MNRAS.345.1057M,2004A&A...414..895F}. However, there are some sources following a steeper R-X correlation with a power-law index of $\sim$1.4 at the high $L_\mathrm{X}$, a flat R-X correlation with a power-law index of $\sim$0 at the moderate $L_\mathrm{X}$, and the standard R-X correlation with a powerlaw index $\sim$0.6 at low luminosity \citep[e.g.][]{2011MNRAS.414..677C,2014ApJ...788...52C,2016MNRAS.463.2287X}. The physical reason of this ``hybrid" R-X correlation is still under debate. The slope change of the correlation may attribute to the different accretion modes or jet physics at low and high luminosity\citep{2016MNRAS.456.4377X,2018MNRAS.481.4513I,2018MNRAS.473.4122E}. The flat R-X correlation of Mrk~1018 is much shallower than the correlation of a sample of black holes in \citet{2012MNRAS.419..267P}. \textcolor{red}{It implies that the radio variability timescale is much longer than X-ray or Mrk~1018 is accreting in the different mode from the samples. }


