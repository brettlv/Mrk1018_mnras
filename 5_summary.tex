\section{Summary}
\label{sec:conclusion}
The main results are summarized as follows,
\begin{enumerate}

\item We present the long-term and multi-wavelength variabilities for the CL-AGN of Mrk 1018. We find that optical/UV emission start to decline in type I case even the X-ray emission is roughly unchanged before a rapid decay to type 1.9 during 2010-2015. We also find a re-flare in both optical and X-ray bands during the decay.   

\item The negative correlations of $\Gamma$-$L_\mathrm{X}/L_\mathrm{Edd}$ and \alphaox-$L_\mathrm{X}/L_\mathrm{Edd}$ in type 1.9 are consisted with the prediction of the radiatively inefficient accretion phase (e.g., ADAF). The Type I case evidently deviate from the negative correlations even the data are still limited, which may stay in a radiatively efficient SSD/corona phase. Therefore, the change of broad emission lines is most possibly regulated by underlying accretion mode. 

\item The radio emission in CL-AGN of Mrk~1018 is roughly unchanged during the changing look, even it is slightly declined about $\sim$ 20\% from 2015 to 2017. Therefore, radio--X-ray correlation is quite flat in Mrk 1018, which is much shallower than that found in low-luminosity AGNs and low-hard state XRBs with a slope of $\sim 0.6$. 

\end{enumerate}
 Further simultaneous observations in multi-wavelength bands are required to understand the evolution of AGN types with spectra and luminosity. 
 