\begin{abstract}
%The physical mechanism for triggering the changing-look phenomenon in active galactic nuclus (AGN) is still unclear. 
We study the long-term multi-wavelength variabilities of a changing-look Seyfert galaxy Mrk~1018, including X-ray, ultraviolet (UV), optical and radio bands. \textcolor{red}{We find that optical/UV emission slightly decrease while 2-10 keV X-ray emission keep roughly unchanged during the type 1 phase before 2010. Both the optical and the X-ray emission suffer rapid decay during 2010-2015 with an \textit{e}-folding timescale of $\sim$ 1000 days, where a re-flare appeared in both optical/UV and X-ray bands. Then Mrk~1018 has already became type 1.9 after 2015.} Both X-ray photon index ($\Gamma$) and the optical-to-X-ray spectral index (\alphaox\,) are anti-correlated with the Eddington scaled 2-10~ keV X-ray luminosity ($L_\mathrm{X}/L_\mathrm{Edd}$) in the type 1.9 phase, which is consistent with the prediction of advection dominated accretion flow (ADAF). However, the type 1 phase apparently does not follow these two anti-correlations, which may have different accretion mode (e.g., standard accretion disk). \textcolor{red}{Our results support that the change of broad emission lines should be directly regulated by the evolution of accretion mode, where the inner cold disk may evaporated into the hot ADAF after the decay. The radio luminosity and X-ray luminosity follow a roughly flat correlation for Mrk 1018, which is much shallower than that found in low-luminosity AGNs and low-hard state X-ray binaries with a slope of $\sim 0.6$.}
\end{abstract}
%% Keywords should appear after the \end{abstract} command. 
%% See the online documentation for the full list of available subject
%% keywords and the rules for their use.
%% \keywords{galaxies: active – galaxies: individual: Mrk 1018 – galaxies: Seyfert}

\begin{keywords}
galaxies: active -- galaxies: Seyfert -- individual: Mrk 1018 
\end{keywords}