\begin{abstract}
The physical mechanism for triggering the changing-look phenomenon in active galactic nuclei (AGNs) is still unclear. We explore this issue based on the multi-wavelength spectral and flux variations for a changing-look AGN Mrk~1018 with long-term observations in the X-ray, optical/ultraviolet(UV), and radio bands. %We find that the optical/UV emission slightly decreases while the X-ray emission keeps roughly unchanged in type 1 phase before 2010. 
Both the optical and the X-ray emission experience rapid decay in changing-look phase during 2010--2015, where a re-flare appears in the optical/UV and X-ray bands. The 5 GHz radio flux decreases by $\sim 20$\% in type 1.9 phase during 2016--2017. We find both X-ray photon index ($\Gamma$) and the optical-to-X-ray spectral index (\alphaox\,) are anti-correlated with the Eddington scaled 2--10~keV X-ray luminosity ($L_\mathrm{X}/L_\mathrm{Edd}$) in the type 1.9 phase. However, the type 1 phase deviates from these two anti-correlations, which suggests that the change of broad emission lines might be regulated by the evolution of accretion disk (e.g., disappearing of the inner cold disk in the type 1.9 phase). %The radio and X-ray luminosity follow a roughly flat correlation for Mrk 1018, which is much shallower than that found in low-luminosity AGNs and low-hard state X-ray binaries.
\end{abstract}

\begin{keywords}
galaxies: active -- galaxies: Seyfert -- individual: Mrk 1018 
\end{keywords}